\documentclass{ar2rc}

\usepackage{url}
\usepackage{siunitx}
\usepackage{natbib}

\usepackage[colorinlistoftodos]{todonotes}
\usepackage{color}
\newcommand{\jdg}[1]{\todo[color=red!30]{#1}}
\newcommand{\all}[1]{\todo[color=blue!30]{#1}}

% better lists
\usepackage{enumitem}
\setlist{nosep}

\hypersetup{hidelinks}

\title{How unprecedented was the February 2021 Texas cold snap?}
\author{James Doss-Gollin, David J. Farnham, Upmanu Lall, and Vijay Modi}
\journal{Environmental Research Letters}

\begin{document}

\maketitle

\listoftodos

We thank the editors and reviewers for their thoughtful, constructive, and timely comments on our initial submission.
In this document we address the specific comments made by the two referees.

\section{Referee \#1}

\RC{
  This is an excellent, timely analysis.
  The authors are right to point to the need for more in depth studies on how cascading failures occurred during the 2021 Texas freeze, and the steps required to minimize their impacts in the future.
  But this is a much needed, early contribution, and very well done.
  I don't have any major criticisms of the work, which is well organized and rigorous.
}

\AR{
  Thank you for your comments.
  We agree that this points to a need for more in depth studies, particularly related to cascading failures of natural gas, electricity systems, water, transportation, and other critical infrastructure systems.
}

\subsection{Some minor comments:}

\RC{
  Considering I think this paper will be widely read and cited, one general comment is that I felt like the citations were a little light in the introduction, especially regarding the 2021 freeze itself.
}

\AR {
  This is a point that reviewer \#2 has also brought up.
  We have added several citations to the introduction, some of which are ERCOT reports released after our initial submission.
  We have also added some references to the discussion.
  Below we list the new sources referenced in this revision.
}
\begin{quote}
  Additional works cited
  \begin{itemize}
    \item \citet{ceser_winterupdate:2021}
    \item \citet{clack_uri:2021}
    \item \citet{dobbins_blackoutdisparity:2021}
    \item \citet{ercotpublic_sarawinter:2020}
    \item \citet{ercotpublic_outagesv2:2021}
    \item \citet{gunter_fishes:1951}
    \item \citet{magness_review:2021}
    \item \citet{mulcahy_urideath:2021}
    \item \citet{nerc_operation:1989}
    \item \citet{nerc_previous:2013}
    \item \citet{nielsen-gammon_txacc:2011}
    \item \citet{osborne_twofreezes:2021}
    \item \citet{romanowsky_stratosphere:2019}
    \item \citet{smead_eyesoftx:2021}

  \end{itemize}
\end{quote}

\RC {
  `Texas interconnect' (which I assume is the Texas Interconnection bulk electric power system administered by ERCOT) is used several times but (I don't think) ever defined; many folks will not know what this refers to.
  If you added a brief definition it would help.
  If this does refer to the grid, then an additional comment is that I've never heard it referred to as the Interconnect, but rather the Interconnection.
}

\AR{
  Thanks for the helpful clarification.
  Interconnection is a better term and we have revised it throughout.
  We also revise the first sentence of the ``data and methods'' section:
}
\begin{quote}
  We use three distinct datasets to analyze temperature minima in \DIFdelbegin \DIFdel{Texas }\DIFdelend \DIFaddbegin \DIFadd{the region covered by the Texas Interconnection bulk electric power system administered by the Electric Reliability Council of Texas (hence ``Texas Interconnection'') }\DIFaddend through the lens of distributed (each grid cell analyzed separately) and aggregated (weighted averages taken across space) extreme values analysis.
\end{quote}

\RC{
  Regarding the definition of a HDD/CDD: More frequently 65 degrees is used, I think.
}

\AR{
  Thank you for this suggestion.
  It is correct that the EIA uses \SI{65}{\degree F} (\url{https://www.eia.gov/energyexplained/units-and-calculators/degree-days.php}).
  We have re-run our analysis using 65F.
}

\RC{
  On the difficulty of estimating supply side risk -- is it possible at this point to map the locations of plants that lost functionality?
}

\AR{
  This is a helpful suggestion.
  ERCOT has released data on generator outages at \url{http://www.ercot.com/content/wcm/lists/226521/Unit_Outage_Data_20210304__Public_.xlsx}.
  However, this information does not contain longitude or latitude coordinates, and this data set does not have the keys or IDs necessary to cross-reference to the IEA data that we use to plot generation locations.
  Further, this data contains only the theoretical seasonal maximum capacity and the available capacity; the difference between these is not the same as capacity ``lost output due to outages and derates,'' which is summarized in ERCOT documents but is not available in a spatial dataset that would permit mapping.
  An informal examination of the ERCOT dataset referenced above reveals that outages/derates were widespread across the state.
  However, since we have referenced ERCOT and NERC documents on this and previous power outages, we have added some text on supply side risk as shown below.
}
\begin{quote}
  A proximate cause of load shedding in the Texas \DIFdelbegin \DIFdel{Interconnect }\DIFdelend \DIFaddbegin \DIFadd{Interconnection }\DIFaddend during February 2021 was the vulnerability of the electricity generation system to the severe cold temperatures \citep{everhart_iea:2021}.
  Figs 3 and S3 show that this happened even though most parts of the state had previously experienced similarly intense cold, notably in 1989.
  \DIFaddbegin \DIFadd{Yet despite temperatures that were, in aggregate, more intense, the electricity system experienced fewer than three hours of rolling blackouts from December 21-23, 1989 \mbox{%DIFAUXCMD
      \citep{nerc_operation:1989,osborne_twofreezes:2021}}\hspace{0pt}%DIFAUXCMD
    .
    Following the 1983, 1989, and 2011 cold snaps, the North American Electric Reliability Corporation (NERC) warned that ``constraints on natural gas fuel supplies to generating plants'' and ``generating unit trips, derates, or failure to start due to weather related causes'' were common themes across severe and moderate cold extremes \mbox{%DIFAUXCMD
      \citep{nerc_previous:2013}}\hspace{0pt}%DIFAUXCMD
    , foreshadowing many of the causes of February 2021 energy system failures identified by ERCOT \mbox{%DIFAUXCMD
      \citep{ercotpublic_outagesv2:2021,magness_review:2021}}\hspace{0pt}%DIFAUXCMD
    .
  }\DIFaddend While our analysis neglects other meteorological factors, like freezing rain, that may have impeded operations at specific facilities, \DIFdelbegin \DIFdel{our analysis suggests }\DIFdelend \DIFaddbegin \DIFadd{we find }\DIFaddend that the February 2021 failures of energy and electricity systems in the Texas \DIFdelbegin \DIFdel{Interconnect }\DIFdelend \DIFaddbegin \DIFadd{Interconnection }\DIFaddend took place during temperatures \DIFdelbegin \DIFdel{comparable to those previously recorded.
    Similarly, water mains broke in places like Houston where temperatures did not exceed 100-year return levels, underscoring characteristic vulnerability of critical infrastructure systems \mbox{%DIFAUXCMD
      \citep{chester_reliable:2020}}\hspace{0pt}%DIFAUXCMD
  }\DIFdelend \DIFaddbegin \DIFadd{with precedent in the historical record}\DIFaddend .
\end{quote}

\RC{
  In the discussion, you mention the 1983 event as being one of comparable severity, and then talk about the grid's failure in 2021.
  I don't know if this was your intention, but  when I read this sentence the first time I took it to mean the grid is *more* vulnerable now.
  I think you could reasonably make that case with greater electrification of heating.
  But I also wondered about coal-to-gas switching, and whether 1983 would have likely been a different fleet of power plants that were perhaps less likely to lose useable capacity during cold weather.
}

\AR{
  Thank you for the interesting and thought-provoking comment.
  We have added some historical context regarding the 1983 and 1989 events to the discussion section as shown in response to the previous comment.
  However, we have refrained from speculating about the degree to which different technologies might have performed in 2021 since nearly all technologies used in Texas can work in cold temperatures if sufficiently prepared.
}

\section{Referee \#2}

\RC{
  This paper investigates whether the Texas deep freeze event was a black swan event, or if it should have been predicted from previous cold snaps.
  The authors compute the population weighted temperature excursion below \SI{68}{\degree F} as a proxy for heating demand and use that to determine aggregate heating demand for 2021.
  Overall the authors find that demand for previous storms was more intense than the 2021 disaster.
}

\AR{
  Thank you for this helpful summary.
  As we note below, we have removed reference to the term ``black swan'' in favor of language with more specific definitions.
}

\RC{
  The introduction is seriously lacking in citations and references.
  For example, this line ``While polar air excursions are not unusual, they do not typically reach as far south as the Mexican border.''
  And a number of others need citations and would benefit from concrete numbers.
  A question that arises is how many storms have reached the Mexican border in the last 10 years?
  Line 45-46 on page 1 also needs a citation.
}

\AR{
  Thank you for the helpful suggestion to include further references.
  We note that referee \#1 made a similar comment and have provided an annotated list of additional works cited in our response to referee \#1 (see our respose to the first minor comment).
  With specific reference to the lines mentioned, we have removed the reference to the Mexican border from our introductory paragraph.
}
\begin{quote}
  Between February 14th and 17th, 2021, a  northern air mass blanketed much of the continental United States, causing anomalously low surface temperatures across the Great Plains.
  \DIFdelbegin \DIFdel{While polar air excursions are not unusual, they do not typically reach as far south as the Mexican border.
  }\DIFdelend
\end{quote}

\AR{
  We have also clarified the sentence which was on Line 45-46 in the previous submission
}
\begin{quote}
  \DIFdelbegin \DIFdel{It is well documented that Texas has previously experienced severe cold , most }\DIFdelend \DIFaddbegin \DIFadd{Texas state climatologist John Nielsen-Gammon wrote in 2011 that ``winter weather is a danger to TX in part because it is so rare,'' \mbox{%DIFAUXCMD
      \cite{nielsen-gammon_txacc:2011}}\hspace{0pt}%DIFAUXCMD
    .
    Previous cold snaps in Texas, }\DIFaddend notably in 1899, 1951, 1983, 1989, and \DIFdelbegin \DIFdel{2011.
  }\DIFdelend \DIFaddbegin \DIFadd{2011 (see \mbox{%DIFAUXCMD
      figs 1 and S1}\hspace{0pt}%DIFAUXCMD
    ), have affected both human and ecological systems.
  }
\end{quote}

\RC{
  In the introduction the ``population weighted temperature excursion below \SI{68}{\degree F}'' needs a definition for temperature excursion.
  Figure 1 is missing citations for the data.
  What do you mean by days ending?
  Were these temperatures recorded at a specific time of day?
  Figure 1 would be improved by adding column headers for the averages (1-day, 3 day, and 5 day).
}

\AR{
  Thank you for calling to our attention these unclear points.
  First, we have changed the subplot titles of Figure 1 from ``N days ending YYYY-MM-DD'' to ``YYYY-MM-DD to YYYY-MM-DD'' for clarity.
  We have supplemented this with column titles.
  We have also added a reference to the ERA-5 data in the caption.\jdg{update the figure as described}
}
\begin{quote}
  TODO
\end{quote}
\AR{
  Finally, we replaced the confusing term ``temperature excursion'' with the more precise definition below (noting that referee \#1 suggests to use \SI{65}{\degree F} rather than \SI{68}{\degree F}).
}
\begin{quote}
  To answer this question, we first compute the population weighted \DIFdelbegin \DIFdel{temperature excursion below \mbox{%DIFAUXCMD
      \SI{68}{\degree F} }\hspace{0pt}%DIFAUXCMD
  }\DIFdelend \DIFaddbegin \DIFadd{difference between observed temperatures and a standard indoor temperature of \mbox{%DIFAUXCMD
      \SI{65}{\degree F} }\hspace{0pt}%DIFAUXCMD
  }\DIFaddend as a proxy for the unknown heating demand\ldots
\end{quote}

\RC{
  A weakness of this paper is the lack of comment about the capacity and size of the Texas grid.
  How does the inferred demand compare to the size of the ERCOT power grid and the interconnection capability over time? We know that the ERCOT grid is larger in 2021 than in 1989, but from Figure 2 there should have been more demand in 1989 than in 2021.
  Would most of the people have wood stoves?
  Was the blackout and number of people impacted as large in the past, as it was in 2021?
}

\AR{
  Thank you for bringing this point to our attention.
  To minimize confusion, we have clarified our terminology so that ``inferred heating demand'' is not ``inferred heating demand per capita.''
  This is reflected throughout the document.
  However, comparing the actual infrastructure failures across different events (such as 1989) would require understanding and modeling cascading failures of interdependent infrastructure systems, which is beyond the scope of our analysis.
  Instead, we have cited ERCOT's estimates of what peak loads would have been without load shedding \citep[\SI{76819}{\mega\watt};][]{magness_review:2021}, ERCOT reports \citep{ercotpublic_outagesv2:2021,ercotpublic_sarawinter:2020}, and NERC reports \citep{nerc_previous:2013,nerc_operation:1989} that shed light on impacts of previous storms.
}

\RC{
  On page 7 line 51 the authors say the Oklahoma recorded some water main breaks and rolling blackouts, but infrastructure impacts were less severe.
  To strengthen this paper, the authors should provide insights as to why this was the case for OK, and not TX.
}

\AR{
  Thank you for identifying this vague statement.
  We do not wish to speculate about the preparedness of water systems.
  Instead, we cite \citet{ceser_winterupdate:2021}'s description of the storm's impact on Southwest Power Pool (SPP) and Midcontinent Independent System Operator (MISO), and on their customers
}
\begin{quote}
  Outside the Texas \DIFdelbegin \DIFdel{Interconnect region, much of the }\DIFdelend \DIFaddbegin \DIFadd{Interconnection region, the Midcontinent Independent System Operator and Southwest Power Pool instructed utilities to shed firm load, but of the approximately 4.89 million customers without power in Texas, Louisiana, and Oklahoma, an estimated 4.5 million were in }\DIFaddend Texas \DIFdelbegin \DIFdel{Panhandle and central Oklahomaexperienced intense cold with a return period greater than 100 years at multiple durations.
    Although Oklahoma recorded some water main breaks \mbox{%DIFAUXCMD
      \citep{crum_water:2021} }\hspace{0pt}%DIFAUXCMD
    and rolling blackouts for several hours \mbox{%DIFAUXCMD
      \citep{money_oklahoma:2021}}\hspace{0pt}%DIFAUXCMD
    , infrastructure impacts were less severe than in Texas despite recording more extreme (relative both to its own historical record and in absolute terms) temperatures}\DIFdelend \DIFaddbegin \DIFadd{\mbox{%DIFAUXCMD
      \citep{ceser_winterupdate:2021}}\hspace{0pt}%DIFAUXCMD
  }\DIFaddend .
\end{quote}

\RC I understand that a big takeaway from the paper is that the cold temperatures and heat demand could be predicted since there were similar demand numbers from 1989, but one year of similar demand does not eliminate this as a black swan event. From the discussion period it seems that over a 30-year time period this type of event has only happened twice. That seems very rare in my opinion.

\AR{
  Thank you for this helpful comment.
  We have clarified our language to more clearly reflect our research question, which is to what extent the temperatures were ``unprecedented'' or ``known to be possible.''
  We also use the term ``extreme'' within the narrow context of addressing questions like ``how extreme was the event'' using statistical extreme value theory, which provides a specific meaning.
  To ensure clarity and reduce the possibility of misinterpretation, we have removed terms like ``rare,'' ``black swan,'' or ``surprising'' that do not have specific interpretations, as well as  the phrase ``could have been anticipated,'' which could mean either ``known to be possible'' or ``had been forecasted''.
  For example, we have clarified language in the abstract, introduction, and discussion:
}
\begin{quote}
  This motivates the question: \DIFdelbegin \DIFdel{was the cold }\DIFdelend \DIFaddbegin \DIFadd{were the temperatures }\DIFaddend that contributed to this infrastructure failure \DIFdelbegin \DIFdel{a ``black swan'' that could not have been anticipated}\DIFdelend \DIFaddbegin \DIFadd{beyond what was known to be possible}\DIFaddend , or did historical storms \DIFdelbegin \DIFdel{provide }\DIFdelend \DIFaddbegin \DIFadd{offer }\DIFaddend a precedent?
\end{quote}
\begin{quote}
  It is therefore important to assess whether \DIFdelbegin \DIFdel{the February 2021 cold snap could have been anticipated.
  }\DIFdelend \DIFaddbegin \DIFadd{historical data offered a precedent for the temperatures observed during February 2021.
  }\DIFaddend
\end{quote}
\begin{quote}
  Our analysis quantifies the \DIFdelbegin \DIFdel{probability with which temperature extremes }\DIFdelend \DIFaddbegin \DIFadd{frequency with which the temperatures observed during February 2021 }\DIFaddend could have been \DIFdelbegin \DIFdel{anticipated }\DIFdelend \DIFaddbegin \DIFadd{expected to occur }\DIFaddend \emph{a priori}.
\end{quote}

\RC The conclusions says that several other storms would have resulted in similar demand. How many?

\AR{
  Thank you for encouraging us to be more specific with our claims.
  We instead identify four storms with Inferred Per Capita Demand for Heating within 90\% that of February 2021.
  The conclusion now reads as follows
}
\begin{quote}
  The February 2021 cold snap was the most intense in 30 years, but was not without precedent in the full historical record.
  In addition to the record cold conditions of 1899 (fig.~S1), we estimate that the weather of December 1989 would have resulted in \DIFdelbegin \DIFdel{slightly }\DIFdelend higher 6-hour and 2-day aggregate \DIFdelbegin \DIFdel{heating demands }\DIFdelend \DIFaddbegin \DIFadd{per capita demand for heating }\DIFaddend over the Texas \DIFdelbegin \DIFdel{Interconnect }\DIFdelend \DIFaddbegin \DIFadd{Interconnection than the February 2021 event }\DIFaddend had it occurred \DIFdelbegin \DIFdel{in February 2021.
    Several other storms since 1950 would have produced nearly as much }\DIFdelend \DIFaddbegin \DIFadd{with today's population.
    Storms in February 1951, January 1962, and December 1983 would have resulted in at least 90\% as much per capita }\DIFaddend demand for heating \DIFaddbegin \DIFadd{had they occurred today}\DIFaddend .
  Given upward trends in the electrification of heating, it is likely that future cold snaps will cause peak annual loads on the Texas \DIFdelbegin \DIFdel{Interconnect }\DIFdelend \DIFaddbegin \DIFadd{Interconnection }\DIFaddend to occur during the winter season.
  Infrastructure expansion necessitated by a rapidly growing population offers Texas the opportunity to invest in a more resilient energy system.
\end{quote}

\RC{
  This paper needs to tie in the size and capacity of the Texas grid to the demand projections.
  Without it I believe the claims are overstated.
}

\AR{
  Thank you for this suggestion.
  The first step we have taken is, as noted above, to refer to ``inferred demand for heating per capita'' to emphasize that we are not accounting for historic population change.
  Beyond population, changes in such factors as the fraction of heating that is electrified, the distribution of building sizes, and the degree of insulation in buildings will affect total electricity demand.
  Turning to the future, large uncertainties in factors like population and technology (for example, electric vehicles and battery storage) will affect electricity demand during cold snaps.
  However, modeling these changes and their implications is beyond the scope of this analysis.
  The second step we have taken is to cite data released by ERCOT on power demand and load shed during the February 2021 storm, as described above, in both the introduction and discussion.
}

\bibliographystyle{agufull}
\bibliography{references}

\end{document}
